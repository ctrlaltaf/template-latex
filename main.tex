\documentclass[12pt]{article}

% ----------------------
% Packages
% ----------------------
\usepackage{graphicx}   % for figures
\usepackage{amsmath, amssymb} % math symbols
\usepackage{hyperref}   % hyperlinks
\usepackage{float}      % for [H] float option if needed

% ----------------------
% Document info
% ----------------------
\title{Example LaTeX + Python Project}
\author{Your Name}
\date{\today}

\begin{document}
\maketitle

% ----------------------
% Sections
% ----------------------
\section{Introduction}
This is an example project where we combine \LaTeX{} for writing 
and Python (via Jupyter) for generating figures and results.

\section{Methods}
We used Python (see \texttt{notebooks/analysis.ipynb}) to generate 
the plots. Figures are saved in the \texttt{figs/} folder.

\section{Results}
Here is a sine curve generated by Python:

\begin{figure}[htbp]
    \centering
    \includegraphics[width=0.8\linewidth]{figs/sine_curve.pdf}
    \caption{A sine curve plotted using Python and included in \LaTeX{}.}
    \label{fig:sine}
\end{figure}

As shown in \cite{example2020}, reproducibility improves 
when we combine code and writing in one workflow.

\section{Discussion}
This setup makes it easy to keep code, analysis, and writing 
in one project. You just re-run the Jupyter notebook to update 
figures, then recompile this \LaTeX{} document.

% ----------------------
% References
% ----------------------
\bibliographystyle{plain}
\bibliography{references}

\end{document}